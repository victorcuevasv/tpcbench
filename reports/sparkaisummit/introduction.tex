\section{Introduction}\label{introduction}

The current business environment requires modern enterprises to adopt and continuously adapt big data analytics in a prompt and cost-effective manner. In this context, SQL-based data analytics is of foremost importance, due to its ability to generate concrete and clear results in a flexible and declarative way.

In turn, such large-scale data analytics efforts require considerable computational resources, whose upfront acquisition is prohibitive for most small to medium businesses. Cloud computing services aim to provide an economical alternative, in which users are granted computational resources dynamically at will, paying only proportionally to the time of their use.

Relying to a large extent on free and open-source software, sophisticated data analytics services have been built on top of these cloud-based computational infrastructures. In some cases, they are developed and supported by the same cloud provider company, while in others third parties manage the software and services. Amazon Web Services EMR Presto and EMR Spark are examples of the former while the Databricks Unified Analytics Platform represents the latter.

We present in this report a comparison of the SQL engines provided by these cloud services based on the implementation, execution and analysis of the TPC-DS Benchmark. Our goal is to help users to determine which system will be more cost-effective for their needs, and which advantages and disadvantages they have to take into consideration.




